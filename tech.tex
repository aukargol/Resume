\documentclass[11pt,a4paper,sans]{moderncv}

\moderncvstyle{classic}    
\moderncvcolor{blue}  
\usepackage[scale=0.75]{geometry}

\usepackage{setspace}
\usepackage[backend=bibtex, style=ieee]{biblatex}
\bibliography{abbrev}
\usepackage[english]{babel}

\firstname{Agata U.}
\familyname{Kargol}
\title{Curriculum Vitae}                
\address{715 Harvard Avenue}{St. Louis MO 63130}
\mobile{+1~(504)~481~1373}                          
\email{aukargol@gmail.com}
\email{aukargol@wustl.edu}                               
\homepage{http://www.cse.wustl.edu/\textasciitilde{}kargola/} 
\extrainfo{Citizenship Status: \textbf{US Citizen} }                       
                
%\photo[64pt][0.4pt]{example-image-a}              
                       
\renewcommand*{\cventry}[7][.25em]{%
  \cvitem[#1]{#2}{%
    {\bfseries#3}%
    \ifthenelse{\equal{#4}{}}{}{ \hfill{\slshape#4}}%
    \ifthenelse{\equal{#5}{}}{}{ #5}%
    \ifthenelse{\equal{#6}{}}{}{ #6}%
    \strut%
    \ifx&#7&%
      \else{\newline{}\begin{minipage}[t]{\linewidth}\small#7\end{minipage}}\fi}}

\begin{document}
\makecvtitle

\vspace{-4em}
\section{Education}
\cventry{2011--2015}{Master of Science in Computer Science}{}{}{\hfill 3.46/4}{\textit{Washington University in St. Louis}, St. Louis, MO \\ \textbf{Awards:} 2012 NSF Graduate Research Fellowship Recipient}
\cventry{2007--2011}{Bachelor of Science in Computer Science}{}{}{\hfill 3.615/4}{\textit{University of Alabama}, Tuscaloosa, AL \\ \textbf{Honors:} \textit{cum laude}, University Honors Program, International Honors Program, Computer-Based Honors Program, University Fellows Experience, Dean's List \\ \textbf{Minors:} Mathematics, Computer-Based Honors Program, Telecommunication and Film}

\vspace{-0.7em}
\section{Languages}
\cvitem{Fluent}{English, Polish}

\vspace{-0.7em}
\section{Programming Languages}
\cvitem{Proficient}{Python, Matlab, JavaScript,  HTML, CSS, C++, C, Bash}
\cvitem{Familiar}{Perl, C\#, Haskell, SmallTalk, VHDL, AVR, MIPS, mySQL}

\vspace{-0.7em}
\section{Technical Skills}
\cvitem{Frameworks}{ Amazon Mechanical Turk Framework, Django, Qt}
\cvitem{Operating Systems}{ GNU/Linux (8 years), Windows (16 years), Mac OS X (10 years), Robot Operating System (ROS, 2 years)}
\cvitem{Software}{ \LaTeX, Matlab, Microsoft Office Suite, Adobe Photoshop, Apple Final Cut Pro}
\cvitem{Certifications}{ Apple Final Cut Pro 5 and 6}

\vspace{-0.7em}
\section{Professional Experience}
\cventry{2012-2015}{Student Researcher, Dr. Robert Pless}{Washington Univ. in St. Louis}{}{}{\small Worked with the Archive of Many Outdoor Scenes (AMOS), the world's largest archive of public outdoor webcams, which contains over 500 million images. \vspace{0.3em} \\
%\cvitem[-1em]{}{
\normalsize \textit{AMOS Development}
\vspace{0.2em}
\small
\begin{itemize}
	\item Built and managed AMOS crowdsourced tasks with Amazon Mechanical Turk
	\item Built Django tools for creation and management of Mechanical Turk tasks
	\item Developed backend features within AMOS's Django framework as part of a team
	\item Required Python, JavaScript, HTML, CSS, Amazon Mechanical Turk, Django
\end{itemize} \vspace{0.3em}
%}
%\cvitem[-1em]{}{ 
\normalsize \textit{Pedestrian Detection in Webcam Imagery for Public Health Applications} 
\vspace{0.2em}
\small
\begin{itemize}
	\item Explored effectiveness of existing machine learning tools and related algorithms for pedestrian detection in webcams
	\item Improved detection algorithms through background subtraction and shadow detection methods to overcome webcam-specific problems
	\item Required Python, Matlab, shell scripting, C++
\end{itemize}
}
\cventry{2011-2012}{Student Researcher, Dr. Bill Smart}{Washington Univ. in St. Louis}{}{}{Developed applications for Willow Garage's PR2 robot to allow the PR2 to act a surrogate for a paraplegic user.
\begin{itemize}
	\item Created point-and-click interface for object manipulation, which was designed to be incorporated into a larger software application suite
	\item Learned about robot functionality and limitations, interface design based on user requirements, and the mathematical concepts of robot visual processing
	\item Required ROS, Python, C++, Qt framework, OpenCV
\end{itemize}
}

\cventry{2010-2011}{Student Researcher, Dr. Monica Anderson}{Univ. of Alabama}{}{} {Explored effectiveness of using robots and Python in introductory computer science class. 
\begin{itemize}
	\item Created and supported a simplified API to control iRobot Creates
	\item Helped teach two sections of an introductory computer science class
	\item Required Python
\end{itemize}
}

\vspace{-0.8em}
\section{Awards}
\vspace{-0.2em}
\cvline{2012}{NSF Graduate Research Fellowship Recipient \hfill \textit{Washington Univ. in St. Louis}}
\cvline{2011}{Dept. of Computer Science Capstone Engineering \hfill \textit{Univ. of Alabama} \newline Society Outstanding Senior}
\cvline{2011}{ACM Chapter Outstanding Senior \hfill \textit{Univ. of Alabama }}
\cvline{2010-2011}{NSF STEM Grant for Increasing Diversity in Computing \hfill \textit{Univ. of Alabama} }
\cvline{2007-2011}{Academic Elite Scholarship Recipient \hfill \textit{Univ. of Alabama}}
\cvline{2007-2011}{Drummond Company Gift  Scholarship Recipient \hfill \textit{Univ. of Alabama}}


\vspace{-0.5em}
\nocite{*}
\begingroup
\setstretch{0.5}
\setlength\bibitemsep{0.4em}
\printbibliography[title=Publications]
\endgroup

\vspace{-1em}
\section{Select Courses}
\vspace{-0.2em}
\cvitem{Computer Science}{Advanced Mobile Robotics, Computer Vision, Computational Photography, Computation Geometry, Computer Graphics, Advanced Machine Learning}
\cvitem{Computer Engineering}{Computer Systems Architecture, Digital Logic, Microcomputers, Computer Architecture}
\cvitem{Mathematics}{Calculus 1/2/3, Applied Matrix Theory, Linear Algebra, Discrete Math, Statistical Methods in Analysis, Theory of Probability}



\end{document}

